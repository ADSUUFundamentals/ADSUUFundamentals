%%% Title:    FTDS: Lecture 1
%%% Author:   Kyle M. Lang
%%% Created:  2016-01-28
%%% Modified: 2022-11-09

\documentclass[10pt]{beamer}
\usetheme{Utrecht}

\usepackage{graphicx}
\usepackage[natbibapa]{apacite}
\usepackage[libertine]{newtxmath}
\usepackage{booktabs}
\usepackage{caption}
\usepackage{hyperref}

\newcommand{\rmsc}[1]{\textrm{\textsc{#1}}}
\newcommand{\pkg}[1]{\textbf{#1}}
\newcommand{\code}[1]{\texttt{#1}}

\title{R \& RStudio Basics}
\subtitle{Fundamental Techniques in Data Science}
\author{Kyle M. Lang}
\institute{Department of Methodology \& Statistics\\Utrecht University}
%\date{}

%------------------------------------------------------------------------------%

\begin{document}

\begin{frame}[t, plain]
  \titlepage
\end{frame}

%------------------------------------------------------------------------------%

\begin{frame}{Outline}
  \tableofcontents
\end{frame}

%------------------------------------------------------------------------------%

\begin{frame}{Attribution}
  
This course was originally developed by Gerko Vink. You can access the original
version of these materials on Dr. Vink's GitHub page:
\url{https://github.com/gerkovink/fundamentals}.

Some of the materials in this repository have been modified. Any errors or
inaccuracies introduced via these modifications are fully my own responsibility
and shall not be taken as representing the views and/or beliefs of Dr. Vink.

\href{https://www.gerkovink.com/fundamentals}{www.gerkovink.com/fundamentals}

\end{frame}

%------------------------------------------------------------------------------%

\sectionslide{Open-Source Software}

%------------------------------------------------------------------------------%

\begin{frame}{What is ``Open-Source''?}

  R is an open-source software project, but what does that mean?
  \va
  \begin{itemize}
  \item Source code is freely available to anyone who wants it.
    \vb
    \begin{itemize}
    \item Free Speech, not necessarily Free Beer
    \end{itemize}
    \vb
  \item Anyone can edit the original source code to suit their needs.
    \vb
    \begin{itemize}
    \item Ego-less programming
    \end{itemize}
    \vb
  \item Many open source programs are also ``freeware'' that are available free
    of charge.
    \vb
    \begin{itemize}
    \item R is both open-source and freeware
    \end{itemize}
  \end{itemize}

\end{frame}

%------------------------------------------------------------------------------%

\begin{frame}{Strengths of Open-Source Software}

  \rmsc{Freedom}
  \vb
  \begin{itemize}
  \item If the software you are using is broken (or just limited in capability), 
    you can modify it in any way you like.
    \vb
  \item If you are unsure of what the software you are using is doing, you can 
    dig into the source code and confirm its procedures.
    \vb
  \item If you create some software, you can easily, and independently, 
    distribute it to the world.
    \vb
    \begin{itemize}
    \item There is a global community of potential users that are all linked via 
      a common infrastructure that facilitates open-source software development 
      and distribution.
    \end{itemize}
  \end{itemize}

\end{frame}

%------------------------------------------------------------------------------%

\begin{frame}{Strengths of Open-Source Software}

  \rmsc{Peer Review}
  \vb
  \begin{itemize}
  \item Every user of open-source software is a reviewer of that software.
    \vb
  \item What ``bedroom programmers'' lack in term of quality control procedures 
    is overcome by the scrutiny of a large and empowered user-base.
    \vc
    \begin{itemize}
    \item When we use closed source software, we are forced to trust the honesty 
      of the developing company.
      \vb
    \item We have no way of checking the actual implementation.
    \end{itemize}
  \end{itemize}

\end{frame}

%------------------------------------------------------------------------------%

\begin{frame}{Strengths of Open-Source Software}

  \rmsc{Accessibility}
  \vb
  \begin{itemize}
  \item Many open-source programs (like R) can be downloaded, for free, from the 
    internet.
    \begin{itemize}
      \vb
    \item You can have R installed on all of you computers (and your mobile 
      phone, your car's info-tainment system, your microwave, your clock-radio,
      ...).
      \vb
    \item No need to beg, borrow, or steal funds to get yourself up-and-running 
      with a cutting-edge data analysis suite.
    \end{itemize}
    \vb
  \item Licensing legality is very simple---no worries about being sued for 
    installing open-source software on ``too many'' computers.
    \vb
  \item Open-source software tends to run on more platforms than closed-source 
    software will.
  \end{itemize}

\end{frame}

%------------------------------------------------------------------------------%

\subsection{Open-Source Licensing}

%------------------------------------------------------------------------------%

\begin{frame}{A Note on Licensing}

  Some popular open-source licenses:
  \vb
  \begin{itemize}
  \item The GNU General Public License (GPL)
    \begin{itemize}
    \item \url{http://www.gnu.org/licenses/gpl-3.0.en.html}
    \end{itemize}
    \vc
  \item The GNU Lesser General Public License (L-GPL)
    \begin{itemize}
    \item \url{http://www.gnu.org/licenses/lgpl-3.0.en.html}
    \end{itemize}
    \vc
  \item The Apache License
    \begin{itemize}
    \item \url{http://www.apache.org/licenses/}
    \end{itemize}
    \vc
  \item The BSD 2-Clause License (FreeBSD License)
    \begin{itemize}
    \item \url{http://opensource.org/licenses/BSD-2-Clause}
    \end{itemize}
    \vc
  \item The MIT License
    \begin{itemize}
    \item \url{https://opensource.org/licenses/MIT}
    \end{itemize}
  \end{itemize}

\end{frame}

%------------------------------------------------------------------------------%

\begin{frame}{A Note on Licensing}

  Many open-source licenses (e.g., GPL, L-GPL) ``copyleft'' their products.
  \vb
  \begin{itemize}
  \item Copyleft is designed to ensure that open-source software cannot be 
    closed.
    \vc
    \begin{itemize}
    \item I can't take your copylefted software, repackage it, and sell it in 
      violation of your original licensing terms.
    \end{itemize}
  \end{itemize}
  \va
  Other open-source licenses (e.g., BSD-Types, Apache, MIT) are non-copyleft, 
  ``permissive'' licenses.
  \vb
  \begin{itemize}
  \item Many of these licenses are designed to promote commercialization of 
    open-source products.
    \vc
    \begin{itemize}
    \item E.g., allowing a student to develop a company selling a product they 
      developed for their dissertation
    \end{itemize}
  \end{itemize}
  
\end{frame}

%------------------------------------------------------------------------------%

\sectionslide{The R Statistical Programming Language}

%------------------------------------------------------------------------------%

\subsection{What is R?}

%------------------------------------------------------------------------------%

\begin{frame}{What is R?}
  
  R is a holistic (open-source) software system for data analysis and
  statistical programming.
  \vc
  \begin{itemize}
  \item R is an implementation of the S language.
    \begin{itemize}
    \item Developed by John Chambers and colleagues 
      \begin{itemize}
      \item \citet{beckerChambers:1984}
      \item \citet{beckerEtAl:1988}
      \item \citet{chambersHastie:1992}
      \item \citet{chambers:1998}
      \end{itemize}
    \end{itemize}
    \vc
  \item Introduced by \citet{ihakaGentleman:1996}.
    \begin{itemize}
    \item Currently maintained by the \emph{R Core Team}.
    \end{itemize}
    \vc
  \item Support by thousands of world-wide contributors.  
    \begin{itemize}
    \item Anyone can contribute an R package to the \emph{Comprehensive R 
        Archive Network} (CRAN) 
    \item Must conform to the licensing and packaging requirements.
    \end{itemize}
  \end{itemize}
  
\end{frame}

%------------------------------------------------------------------------------%

\begin{frame}{What is R?}

  I prefer to think about R as a \emph{statistical programming language}, rather 
  than as a data analysis program.
  \vb
  \begin{itemize}
  \item R \textbf{IS NOT} its GUI (no matter which GUI you use).
    \vb
  \item You can write R code in whatever program you like (e.g., RStudio, EMACS, 
    VIM, Notepad, directly in the console/shell/command line).
    \vb
  \item R can be used for basic (or advanced) data analysis, but its real 
    strength is its flexible programming framework.
    \vc
    \begin{itemize}
      \item Tedious tasks can be automated.
        \vc
      \item Computationally demanding jobs can be run in parallel.
        \vc
      \item R-based research \emph{wants} to be reproducible.
        \vc
      \item Analyses are automatically documented via their scripts.
    \end{itemize}
  \end{itemize}

\end{frame}

%------------------------------------------------------------------------------%

\begin{frame}{What is RStudio?}

RStudio is an integrated development environment (IDE) for R.

\begin{itemize}
\item Adds a bunch of window dressing to R
\item Also open-source
\item Both free and paid versions
\end{itemize}

R and RStudio are independent entities.

\begin{itemize}
\item You do not need RStudio to work with R.
\item You are analyzing your data with R, not RStudio
  \begin{itemize}
  \item RStudio is just the interface through which you interact with R.
  \end{itemize}
\end{itemize}

\end{frame}

%------------------------------------------------------------------------------%

\begin{frame}{Getting R}

  You can download R, for free, from the following web page:
  \va
  \begin{itemize}
  \item \url{https://www.r-project.org/}
  \end{itemize}
  \va
  Likewise, you can freely download RStudio via the following page:
  \va
  \begin{itemize}
  \item \url{https://www.rstudio.com/}
  \end{itemize}

\end{frame}

%------------------------------------------------------------------------------%

\subsection{Using R}

%------------------------------------------------------------------------------%

\begin{frame}{What to Expect when Opening R}

As noted above, we have many ways of interacting with R:
  \vb
  \begin{itemize}
  \item Base R
    \vb
  \item EMACS
    \vb
  \item RStudio
    \vb
  \item Text-only console (i.e., even more base R)
  \end{itemize}

\end{frame}

%------------------------------------------------------------------------------%

\begin{frame}{How R Works}

  R is an interpreted programming language.
  \vb
  \begin{itemize}
  \item The commands you enter into the R \emph{Console} are executed 
    immediately.
    \vc
  \item You don't need to compile your code before running it.
    \vc
  \item In this sense, interacting with R is similar to interacting with other
    syntax-based statistical packages (e.g., SAS, STATA, Mplus).
  \end{itemize}

\end{frame}

%------------------------------------------------------------------------------%

\begin{frame}{How R Works}
  
  R mixes the \emph{functional} and \emph{object-oriented} programming
  paradigms.
  \vc
  \begin{columns}
    \begin{column}{0.5\textwidth}
      
      \begin{center}
        \rmsc{Functional}
      \end{center}
      
      \begin{itemize}
      \item R is designed to break down problems into functions.
        \vc
      \item Every R function is a first-class object.
        \vc
      \item R uses pass-by-value semantics.
      \end{itemize}
      
    \end{column}
    \begin{column}{0.5\textwidth}
      
      \begin{center}
        \rmsc{Object-Oriented}
      \end{center}
      
      \begin{itemize}
      \item Everything in R is an object.
        \vc
      \item R functions work by creating and modifying R objects.
        \vc
      \item The R workflow is organized by assigning objects to names.
      \end{itemize}
      
    \end{column}
  \end{columns}
  
\end{frame}

%------------------------------------------------------------------------------%

\begin{frame}{Interacting with R}

  When working with R, you will write \emph{scripts} that contain all of the 
  commands you want to execute.
  \va
  \begin{itemize}
    \item There is no ``clicky-box'' Tom-foolery in R.
    \vb
    \item Your script can be run interactively or in ``batch-mode'', as a 
      self-contained program.
  \end{itemize}
  \va
  The primary purpose of the commands in your script will be to create and 
  modify various objects (e.g., datasets, variables, function calls, graphical 
  devices).

\end{frame}

%------------------------------------------------------------------------------%

\begin{frame}{Getting Help}
  
Everything published on the Comprehensive R Archive Network (CRAN), and intended 
for R users, must be accompanied by a help file. 

\begin{itemize}
\item If you know the name of the function (e.g., `anova()`), then execute
  `?anova` or `help(anova)`. 
\item If you do not know the name of the function, type `??` followed by your
  search criterion. For example, `??anova` returns a list of all help pages that
  contain the word "anova". 
\end{itemize}

Alternatively, the internet will tell you almost everything you'd like to know

\begin{itemize}
\item Sites such as http://www.stackoverflow.com and http://www.stackexchange.com 
can be very helpful. 
\item If you google R-related issues, include "R" somewhere in your search string.
\end{itemize}

\end{frame}

%------------------------------------------------------------------------------%

\begin{frame}{Packages}

Packages give R additional functionality. 

\begin{itemize}
\item By default, some packages are included when you install R.
\item These packages allow you to do common statistical analyses and data
  manipulation.
\item Installing additional packages allows you to perform state-of-the-art
  statistical analyses. 
\end{itemize}

These packages are all developed by R users, so the throughput process is very 
timely.

\begin{itemize}
\item Newly developed functions and software are readily available
\item Software implementations of new methods can be quickly dissemintated
\item This efficiency differs from other mainstream software (e.g., SPSS, SAS, 
  MPlus) where new methodology may take years to be implemented. 
\end{itemize}

A list of available packages can be found on \href{https://cran.r-project.org}{CRAN}.

%------------------------------------------------------------------------------%

\begin{frame}{Installing \& Loading Packages}

Install a package (e.g., `mice`):
```{r eval = FALSE}
install.packages("mice")
```

There are two ways to load a package into R
```{r}
library(stats)
require(stats)
```

\end{frame}

%------------------------------------------------------------------------------%

\subsection{Project Management}

%------------------------------------------------------------------------------%

\begin{frame}{Working Directory}

\end{frame}

%------------------------------------------------------------------------------%

\begin{frame}{Directory Structure}

\end{frame}

%------------------------------------------------------------------------------%

\begin{frame}{RStudio Projects}

RStudio projects provide a convenient way to organize all of the code and 
supporting resources for a given research project.

\begin{itemize}
\item Every project has its own history
\item Every research project can have its own RStudio project
\item Every project can have its own directory
\item Every project can have its own version control system
\item R-studio projects can relate to GitHub (or other online) repositories
\end{itemize}

\end{frame}

%------------------------------------------------------------------------------%

\sectionslide{Data I/O}

%------------------------------------------------------------------------------%

\begin{frame}[fragile]{Built-In R Data \& Workspaces}
  
### We have many ways to read data into R

## Load the built-in 'bfi' data from the 'psychTools' package
data(bfi, package = "psychTools")

## Access the documentation for the 'bfi' data
?psychTools::bfi

################################################################################
## PRACTICE PROBLEM 3.1
##
## (a) Use the data() function to load the 'Cars93' dataset from the 'MASS'
##     package.
## (b) Use the dim() function to check the dimensoins of the 'Cars93' data.
##     - How many rows?
##     - How many columns?
##
################################################################################

## Define the directory holding our data
dataDir <- "../data/"

## Load the 'boys' data stored in the R workspace '../data/boys.RData'
load(paste0(dataDir, "boys.RData"))

\end{frame}

%------------------------------------------------------------------------------%

\begin{frame}[fragile]{Delimited Data Types}

## Load the 'diabetes' data stored in tab-delimited file '../data/diabetes.txt'
diabetes <- read.table(paste0(dataDir, "diabetes.txt"),
                       header = TRUE,
                       sep = "\t")

## Load the 2017 UTMB data from the comma-seperated file '../data/utmb_2017.csv'
utmb1 <- read.csv(paste0(dataDir, "utmb_2017.csv"))

### NOTE: For EU-formatted CSV files, use read.csv2()

## Load the 'titanic' data stored in R data set '../data/titanic.rds'
titanic <- readRDS(paste0(dataDir, "titanic.rds"))

################################################################################
## PRACTICE PROBLEM 3.2
##
## (a) Load the dataset saved as '../data/diabetes.rds'.
## (b) Use the str() function to compare the structure of the data you loaded in
##     (a) to the diabetes data loaded above using the read.table() function.
##     - Are there any differences between these two objects? If so, what are
##       the differences?
##
################################################################################

\end{frame}

%------------------------------------------------------------------------------%

\begin{frame}[fragile]{SPSS Data}

### Reading data in from other stats packages can be a bit tricky. If we want to
### read SAV files, the two most popular options are foreign::read.spss() and
### haven::read_spss().

## Use foreign::read.spss() to read '../data/mtcars.sav' into a list
(mtcars1 <- read.spss(paste0(dataDir, "mtcars.sav")))

attributes(mtcars1)

## Read '../data/mtcars.sav' as a data frame
(mtcars2 <- read.spss(paste0(dataDir, "mtcars.sav"), to.data.frame = TRUE))

attributes(mtcars2)

## Read '../data/mtcars.sav' without value labels
(mtcars3 <- read.spss(paste0(dataDir, "mtcars.sav"),
                      to.data.frame = TRUE,
                      use.value.labels = FALSE)
)

attributes(mtcars3)

## Use haven::read_spss() to read '../data/mtcars.sav' into a tibble
(mtcars4 <- read_spss(paste0(dataDir, "mtcars.sav")))

attributes(mtcars4)

## have::read_spss() converts any SPSS variables with labels to labelled vectors
mtcars$am
attributes(mtcars$am)

## Use the labelled::unlabelled() functio to remove the value labels (but not
## the variable labels)
(mtcars5 <- unlabelled(mtcars4))

val_labels(mtcars4)
val_labels(mtcars5)

var_label(mtcars5)

### NOTE: The 'labelled' package provides a bunch of utilities to manipulate the
###       variable and value labels for data coming from, or going to, SPSS.

################################################################################
## PRACTICE PROBLEM 3.3
##
## (a) Use the haven::read_spss() function to load the SPSS dataset saved at
##     '../data/starwars.sav'
## (b) Use the foreign::read.spss() function to load the same dataset as above
##     into a list with variable labels preserved.
## (c) Use the foreign::read.spss() function to load the same dataset as above
##     into a data frame without variable labels.
##
################################################################################

\end{frame}

%------------------------------------------------------------------------------%

\begin{frame}[fragile]{Excel Data}

## Use the readxl::read_excel() function to read the data from the 'titanic'
## sheet of the Excel workbook stored at '../data/example_data.xlsx'
titanic2 <- read_excel(paste0(dataDir, "example_data.xlsx"), sheet = "titanic")

## Use the openxlsx::read.xlsx() function to read the data from the 'titanic'
## sheet of the Excel workbook stored at '../data/example_data.xlsx'
titanic3 <- read.xlsx(paste0(dataDir, "example_data.xlsx"), sheet = "titanic")

str(titanic2)
str(titanic3)

all.equal(as.data.frame(titanic2), titanic3)

################################################################################
## PRACTICE PROBLEM 3.4
##
## (a) Use the openxlsx::read.xlsx() function to load the first 100 rows (not
##     counting column names) of the first 4 columns from the 'diabetes' sheet
##     in the Excel workbook stored at '../data/example_data.xlsx'
## (b) Use the readxl::read_excel() function with an appropriate specification
##     for the 'range' argument to load the chunk of data beginning on Row 3 and
##     Column 2 and ending on Row 100 and Column 7 from the 'titanic' sheet in
##     '../data/example_data.xlsx'
##
################################################################################

\end{frame}

%------------------------------------------------------------------------------%

\subsection{Writing Data}

%------------------------------------------------------------------------------%

\begin{frame}[fragile]{Workspaces \& Delimited Data}

### All of the data reading functions we saw earlier have complementary data
### writing versions.

## The save() function writes an R workspace to disk
save(boys, file = paste0(dataDir, "tmp.RData"))

## For delimited text files and RDS data, the write.table(), write.csv(), and
## saveRDS() function do what you'd expect
write.table(boys,
            paste0(dataDir, "boys.txt"),
            row.names = FALSE,
            sep = "\t",
            na = "-999")
write.csv2(boys, paste0(dataDir, "boys.csv"), row.names = FALSE, na = "")
saveRDS(boys, paste0(dataDir, "boys.rds"))

\end{frame}

%------------------------------------------------------------------------------%

\begin{frame}[fragile]{SPSS Data}

## To write SPSS data, the best option is the haven::write_sav() function.
write_sav(mtcars2, paste0(dataDir, "mctars2.sav"))

## write_sav() will preserve label information provided by factor variables and
## the 'haven_labelled' class, but not by attributes
write_sav(mtcars4, paste0(dataDir, "mctars4.sav"))
write_sav(mtcars5, paste0(dataDir, "mctars5.sav"))

\end{frame}

%------------------------------------------------------------------------------%

\begin{frame}[fragile]{Excel Data}

The 'openxlsx' package provides a powerful toolkit for programmatically
building Excel workbooks in R and saving the results. Of course, it also
works for simple data writing tasks.

## Use the openxlsx::write.xlsx() function to write the 'diabetes' data to an
## XLSX workbook
write.xlsx(diabetes, paste0(dataDir, "diabetes.xlsx"), overwrite = TRUE)

## Use the openxlsx::write.xlsx() function to write each data frame in a list to
## a seperate sheet of an XLSX workbook
write.xlsx(list(titanic = titanic, diabetes = diabetes, mtcars = mtcars),
           paste0(dataDir, "example_data.xlsx"),
           overwrite = TRUE)

         \end{frame}

         % ------------------------------------------------------------------------------%

## Some Programming Tips

- Keep your code tidy

- Use comments (text preceded by `#`) to clarify what you are doing

- When working with functions in RStudio, use the TAB key to quickly access the 
documentation of the function's arguments

- Give your R scripts logical names

- Work with RStudio projects or a consistent directory structure

- If allowed, place your project directories somewhere in the cloud

%------------------------------------------------------------------------------%

## General Style Advice

Use common sense and BE CONSISTENT.

- Browse through the [tidyverse style guide](https://style.tidyverse.org)

  - The point of style guidelines is to enforce a common vocabulary 
  - You want people to concentrate on *what* you're saying, not *how* you're 
  saying it. 

- If the code you add looks drastically different from the extant code, the 
incongruity will confuse readers and collaborators.

Spacing makes your code easier to read and understand

  - `a<-c(1,2,3,4,5)`
  - `a <- c(1, 2, 3, 4, 5)`
  - At least put spaces around assignment operators and after every comma!

%------------------------------------------------------------------------------%

\begin{frame}[allowframebreaks]{References}
  \bibliographystyle{apacite}
  \bibliography{../../bibtex/winter_school_refs.bib}
\end{frame}

%------------------------------------------------------------------------------%

\end{document}
